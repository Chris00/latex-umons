\documentclass[12pt,a4paper]{memoire-umons}
% On peut passer l'option « print » à la classe pour que tous les
% hyperliens (très utiles pour naviguer dans le document à l'écran)
% soient en noir.

\usepackage[utf8]{inputenc}
\usepackage[T1]{fontenc}
\usepackage[french]{babel}
\usepackage{amssymb,amsmath,amsthm}
%\usepackage{times,mathptmx}
\usepackage{biblatex}% Utilise le programme "biber"
\usepackage{hyperref}

\title{Titre}
\author{Prénom \textsc{Nom}}
\date{2020--2021}
\directeur{Nom du directeur}
%\directeurs{Directeur 1\\ Directeur 2}
%\codirecteurs{Co-directeur 1\\Co-directeur 2}
\service{Service dans lequel vous avez fait votre mémoire}
%\discipline{mathématiques}
%\logofac{UMONS_FS}
%\departement{Département de Mathématique}

% Si vous utilisez (conseillé) BibLaTeX pour votre bibliographie :
\addbibresource{memoire.bib}

%%%%%%%%%%%%%%%%%%%%%%%%%%%%%%%%%%%%%%%%%%%%%%%%%%%%%%%%%%%%%%%%%%%%%%%%
%% Vos macros

% \setlength{\overfullrule}{10pt}
% Marque par un carré noir des lignes mal coupées pour pouvoir les
% repérer facilement.


%%%%%%%%%%%%%%%%%%%%%%%%%%%%%%%%%%%%%%%%%%%%%%%%%%%%%%%%%%%%%%%%%%%%%%%%

% Compile uniquement certains morceaux sans perdre les références
% automatiques et la table des matières des parties déjà compilées :
%\includeonly{introduction,chapitre1}

\begin{document}
% Éventuellement utiliser l'environnement « preface » pour avoir une
% numérotation des pages en chiffres romains.
\tableofcontents

\chapter{Introduction}

Ce document est écrit avec l'excellent système \TeX\
\autocite{knuth_texbook_1986}.

La classe «~\texttt{memoire-umons}~» offre un certain nombre de macros
pour votre facilité.  Veuillez lire le manuel.  Voyez aussi la fin du
fichier «~\texttt{memoire-\linebreak[2]umons.cls}~».



%%% Local Variables:
%%% mode: latex
%%% TeX-master: "memoire"
%%% ispell-local-dictionary: "fr"
%%% End:

\chapter{Introduction}

\section{Section 1}

On définit
\begin{equation}
  \label{eq:exp}
  \e^x = \sum_{n=0}^\infty \frac{x^n}{n!}.
\end{equation}


\section{Section 2}



\endinput
%%% Local Variables:
%%% mode: latex
%%% TeX-master: "memoire"
%%% ispell-local-dictionary: "fr"
%%% End:

\chapter{Position du problème}

\section{Section 1}


\section{Section 2}



\endinput
%%% Local Variables:
%%% mode: latex
%%% TeX-master: "memoire"
%%% ispell-local-dictionary: "fr"
%%% End:

% etc


\chapter*{Conclusion}
\addcontentsline{toc}{chapter}{Conclusion}
\markboth{\textsl{Conclusion}}{\textsl{Conclusion}}

Quelques lignes pour finir...

\printbibliography

\end{document}
%%% Local Variables:
%%% mode: latex
%%% TeX-master: t
%%% TeX-PDF-mode: t
%%% ispell-local-dictionary: "fr"
%%% End: 
