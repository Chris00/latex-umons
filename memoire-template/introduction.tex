\chapter{Introduction}

Ce document est écrit avec l'excellent système \TeX\
\autocite{knuth_texbook_1986}.

La classe «~\texttt{memoire-umons}~» offre un certain nombre de macros
pour votre facilité.  Veuillez lire le manuel.  Voyez aussi la fin du
fichier «~\texttt{memoire-\linebreak[2]umons.cls}~».

\bigskip

Le \emph{package} \texttt{hyperef} offre la commande \verb+\autoref+
qui crée un lien en utilisant le contexte de la référence.  Exemples :
\begin{itemize}
\item \verb/\autoref/ pour un chapitre donne : \autoref{ch:prelim} ;
\item \verb/\autoref/ pour une équation donne : \autoref{eq:exp} ;
\item \verb/\autoref/ pour un théorème donne :
  \autoref{thm-fondamental-algebre} ;
\item \verb/\autoref/ pour une proposition donne : \autoref{pos-expo} ;
\item \verb/\autoref/ pour un lemme donne : \autoref{closure} ;
\item \verb/\autoref/ pour une définition donne : \autoref{def:expo} ;
\item \verb/\autoref/ pour une remarque donne :
  \autoref{root-odd-poly}.
\end{itemize}


%%% Local Variables:
%%% mode: latex
%%% TeX-master: "memoire"
%%% ispell-local-dictionary: "fr"
%%% End:
