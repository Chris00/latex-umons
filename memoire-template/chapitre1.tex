\chapter{Préliminaires}
\label{ch:prelim}


\section{Section 1}

On note $\emptyset$ l'ensemble vide.  À ne pas confondre avec
$\epsilon$ qui désigne le mot vide.

\begin{definition}
  \label{def:expo}
  L'exponentielle Népérienne $\e^x$ d'un nombre $x \in \IC$ est
  définie par
  \begin{equation}
    \label{eq:exp}
    \e^x := \sum_{n=0}^\infty \frac{x^n}{n!}.
  \end{equation}
\end{definition}

\begin{proposition}
  \label{pos-expo}
  Pour tout $x \in \IR$, $\e^x > 0$.
\end{proposition}


\section{Section 2}

\begin{lemma}
  \label{closure}
  Soit $A \subseteq \IR$.  On a $A \subseteq \cl A$.
\end{lemma}



\endinput
%%% Local Variables:
%%% mode: latex
%%% TeX-master: "memoire"
%%% ispell-local-dictionary: "fr"
%%% End:
