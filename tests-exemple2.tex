\documentclass[12pt,a4paper, no-auto-maketitle%,correction
]{tests}
% L'option "no-auto-maketitle" assure qu'un test n'est pas
% automatiquement démarré par \begin{document} mais par un \maketitle
% explicite.  Ici, ce \maketitle est généré par \generatetestsfromCSV.

\usepackage[utf8]{inputenc}
\usepackage[T1]{fontenc}
\usepackage{times,mathptmx}
\usepackage[french]{babel}

% Ce fichier CSV sera en général généré par un programme externe.
\begin{filecontents*}{tests-exemple2.csv}
  Nom, Prenom, Section, abstract, Q1
  Küal, Élise, math, 1, 2
  Orval, Übèck, informatique, 2, 1
  Pas d'abstract, &Q1, math, , 1
  Pas de Q1, mais abstract, math, 1
\end{filecontents*}

\exam

\begin{document}
\begin{namequestion}{abstract}
  % L'environnement « namedquestion » peut être utilisé pour capturer
  % un bloc de texte arbitraire.
  \begin{abstract}
    Ceci est le test des mathématiciens.
  \end{abstract}
\end{namequestion}
\begin{namequestion}{abstract}
  \begin{abstract}
    Ceci est le test des informaticiens.
  \end{abstract}
\end{namequestion}

\begin{namequestion}{Q1}
  \begin{question}
    Ceci est la première question (première version).
  \end{question}
\end{namequestion}
\begin{namequestion}{Q1}
  \begin{question}
    Ceci est la première question (seconde version).
  \end{question}
\end{namequestion}


\generatetestsfromCSV{tests-exemple2.csv}{
  \iftestscorrection\else abstract,\fi % Consignes sauf pour la correction
  Q1}

\end{document}
%%% Local Variables:
%%% mode: latex
%%% TeX-master: t
%%% ispell-local-dictionary: "fr"
%%% End:
